\documentclass[a4paper,10pt,spanish]{article}
\usepackage[utf8]{inputenc} % Soporte para los acentos
\usepackage[T1]{fontenc}
\usepackage[spanish]{babel} % Capítulos, seciones, etc. en español
\usepackage[margin=1.5cm]{geometry} % Diseño del documento
\usepackage{multicol} % Escribir doble columna
\usepackage{xcolor} % Usar colores
\usepackage{pstricks}
\usepackage{enumerate} % Cambiar etiquetas de numeración
\usepackage[shortlabels]{enumitem} % Manejo adicional de etiquetas de numeración
\usepackage{graphicx} % Manejo de gráficos y figuras
\graphicspath{ {img/} } %Carpeta que contiene las imagenes
\usepackage{makeidx} % Índice alfabético
% Paquetes adicionales de símbolos matemáticos
\usepackage{amsmath,amssymb,amsfonts,latexsym,cancel} 
\usepackage{newcent} % Fuente New Century Schoolbook
\usepackage{booktabs} % Opciones adicionales para el entorno tabular
\usepackage{longtable} % Para tablas de más de una página
\usepackage{tikz,tkz-tab} % Creación de gráficos
	\usetikzlibrary{matrix,arrows,positioning,shadows,shadings,
					backgrounds,calc,shapes,tikzmark}
                    
\usepackage{tcolorbox,empheq} % Creación de cajas
	\tcbuselibrary{skins,breakable,listings,theorems}
\usepackage{gensymb} % Grados Celcius
\definecolor{green}{RGB}{0,127,0}
\definecolor{yellow}{RGB}{200,200,0}
\pagestyle{myheadings} % Numeración de página en la parte superior

\title{\Huge Procesamiento Digital de Imágenes \\ Parcial 2} % Título
\author{Darién Julián Ramírez \\
		Franco Matzkin \\
        Gianfranco Fagioli} % Autor
\date{\empty} % Fecha

\begin{document}

\maketitle % Mostrar título
\tableofcontents % Tabla de contenidos

\section{Anexo}

%\begin{tcolorbox}[colback=gray!10!white,colframe=black!0!white]
\textbf{Modelo de degradación (frecuencial):}
\begin{align*}
G(u,v) &= H(u,v)F(u,v)
\end{align*}

\begin{quote}
$F(u,v)$: Transformada de Fourier de la imagen sin degradación. \\
$H(u,v)$: Función de degredación. \\
$G(u,v)$: Transformada de Fourier de la imagen degradada. \\
\end{quote}

\textbf{Filtro de media geométrica (frecuencial):}

\begin{align*}
\widehat{F}(u,v) &= \left[\frac{H^{*}(u,v)}{|H(u,v)|^{2}}\right]^{\alpha}
\left[\frac{H^{*}(u,v)}{|H(u,v)|^{2}+\beta\frac{S_{\eta}(u,v)}{S_{f}(u,v)}}\right]^{1-\alpha}
G(u,v)
& \textit{Con }|H(u,v)|^{2} &= H^{*}(u,v)H(u,v)
\end{align*}

\begin{quote}
Espectro de potencia del ruido: $S_{\eta}(u,v)=|N(u,v)|^{2}$ \\
Espectro de potencia de la imagen original: $S_{f}(u,v)=|F(u,v)|^{2}$ \\
\end{quote}

Si $\alpha=1$ entonces,

\begin{align*}
\widehat{F}(u,v) &= \frac{H^{*}(u,v)}{|H(u,v)|^{2}}G(u,v)
& \textit{Remplazando } |H(u,v)|^{2} &= H^{*}(u,v)H(u,v) \\
&= \frac{H^{*}(u,v)}{H^{*}(u,v)H(u,v)}G(u,v)
& \textit{Simplificando.} \\
&= \frac{G(u,v)}{H(u,v)}
& \textit{\textbf{Filtro inverso.}}
\end{align*}

Si $\alpha=0$ entonces,

\begin{align*}
\widehat{F}(u,v) &= \left[\frac{H^{*}(u,v)}{|H(u,v)|^{2}+\beta\frac{S_{\eta}(u,v)}{S_{f}(u,v)}}\right]G(u,v)
& \textit{Reemplazando }H^{*}(u,v) &= \frac{|H(u,v)|^{2}}{H(u,v)} \\
&= \left[\frac{1}{H(u,v)}\frac{|H(u,v)|^{2}}{|H(u,v)|^{2}+\beta\frac{S_{\eta}(u,v)}{S_{f}(u,v)}}\right]G(u,v)
& \textit{\textbf{ Filtro paramétrico de Wiener.}} & \quad 0<\beta<1 \\
\end{align*}

Valores pequeños de $\beta$ restauran mejor la degradación pero no eliminan correctamente el ruido.

Valores altos de $\beta$ restauran pobremente la degradación pero eliminan mejor el ruido.

\begin{align*}
\textit{Si }\beta=0 &\implies \widehat{F}(u,v) = \frac{G(u,v)}{H(u,v)} 
& \textit{\textbf{Filtro inverso.}} \\
\textit{Si }\beta=1 &\implies \widehat{F}(u,v) = \left[\frac{1}{H(u,v)}\frac{|H(u,v)|^{2}}{|H(u,v)|^{2}+\frac{S_{\eta}(u,v)}{S_{f}(u,v)}}\right]G(u,v)
& \textit{\textbf{Filtro de Wiener.}}
\end{align*}

\begin{align*}
\widehat{F}(u,v) &= \left[\frac{1}{H(u,v)}\frac{|H(u,v)|^{2}}{|H(u,v)|^{2}+\frac{\beta}{\rho}}\right]G(u,v) & \rho &= \frac{S_{f}(u,v)}{S_{\eta}(u,v)} & \textit{Relación señal/ruido.}
\end{align*} \\

Cuando no se conocen o no se pueden estimar los espectros de potencia se usa una aproximación:

\begin{align*}
\widehat{F}(u,v) &= \left[\frac{1}{H(u,v)}\frac{|H(u,v)|^{2}}{|H(u,v)|^{2}+K}\right]G(u,v)
\end{align*}

Si $\alpha=\frac{1}{2}$ entonces,

\begin{align*}
\widehat{F}(u,v) &= \left[\frac{H^{*}(u,v)}{|H(u,v)|^{2}}\right]^{\frac{1}{2}}
\left[\frac{H^{*}(u,v)}{|H(u,v)|^{2}+\beta\frac{S_{\eta}(u,v)}{S_{f}(u,v)}}\right]^{\frac{1}{2}} \\
&= \left[\frac{G(u,v)}{H(u,v)} \right]^{\frac{1}{2}}
\left[\frac{1}{H(u,v)}\frac{|H(u,v)|^{2}}{|H(u,v)|^{2}+\beta\frac{S_{\eta}(u,v)}{S_{f}(u,v)}}\right]^{\frac{1}{2}} \\
& \textit{Media geométrica entre el filtro inverso y el filtro paramétrico de Wiener.}
\end{align*}

Si $\alpha=\frac{1}{2}$ y $\beta=1$ entonces se tiene un filtro de ecualización del espectro de potencia.

%\end{tcolorbox}

\end{document}